\documentclass{beamer}
\usetheme{CambridgeUS}
\beamertemplatenavigationsymbolsempty % remove navigation bar

\usepackage{xcolor}
\definecolor{YellowGreen} {HTML}{B5C28C}
\definecolor{ForestGreen} {HTML}{009B55}
\definecolor{MyBackground}{HTML}{F0EDAA}


\usepackage{xltxtra} % load xunicode
\usepackage{polyglossia}
\setmainlanguage{russian}
\setotherlanguage{english}

\let\cyrillicfonttt\monofamily
\newfontfamily\dejaVuSansMono{DejaVu Sans Mono}
% https://github.com/vjpr/monaco-bold/raw/master/MonacoB/MonacoB.otf
\newfontfamily\monacoB{MonacoB}

\usepackage{listings}
\lstdefinestyle{style1}{
  language=haskell,
  numbers=left,
  stepnumber=1,
  numbersep=10pt,
  tabsize=4,
  showspaces=false,
  showstringspaces=false
}
\lstdefinestyle{hsstyle1}
{ language=haskell
         , basicstyle=\monacoB
         , deletekeywords={Int,Float,String,List,Void}
         , breaklines=true
         , columns=fullflexible
         , commentstyle=\color{ForestGreen}
         , escapeinside=§§
         , escapebegin=\begin{russian}\commentfont
         , escapeend=\end{russian}
         , commentstyle=\color{ForestGreen}
         , escapeinside=§§
         , escapebegin=\begin{russian}\monacoB\color{ForestGreen}
         , escapeend=\end{russian}
         , mathescape=true
%          , backgroundcolor = \color{MyBackground}
}

\newcommand{\inline}[1]{\lstinline{haskell}{#1}}
\def\hsinline{\lstinline[style={hsstyle1}]}
\def\inline{\hsinline}

\lstnewenvironment{hslisting} {
    \lstset { style={hsstyle1} }
  }
  {}
  
%%%%%%%%%%%%%%%%%%%%%%%%%%%%%%%%%%%%%%%%%%%%%%%%%%%%%%%%%%  
\setmainfont[
 Ligatures=TeX,
 Extension=.otf,
 BoldFont=cmunbx,
 ItalicFont=cmunti,
 BoldItalicFont=cmunbi,
]{cmunrm}
% С засечками (для заголовков)
\setsansfont[
 Ligatures=TeX,
 Extension=.otf,
 BoldFont=cmunsx,
 ItalicFont=cmunsi,
]{cmunss}
% \setmonofont[Scale=0.6]{Monaco}

\usefonttheme{professionalfonts}
\usepackage{times}
\usepackage{tikz}
\usetikzlibrary{cd}
% \usepackage{tikz-cd}
\usepackage{amsmath}
%\DeclareMathOperator{->}{\rightarrow}
\newcommand\iso{\ensuremath{\cong}}
\usepackage{verbatim}
\usepackage{graphicx}
\usetikzlibrary{arrows,shapes}

\usepackage{fontawesome}
% \newfontfamily{\FA}{Font Awesome 5 Free} % some glyphs missing
\expandafter\def\csname faicon@facebook\endcsname{{\FA\symbol{"F09A}}}
\def\faQuestionSign{{\FA\symbol{"F059}}}
\def\faQuestion{{\FA\symbol{"F128}}}
\def\faExclamation{{\FA\symbol{"F12A}}}
\def\faUploadAlt{{\FA\symbol{"F093}}}
\def\faLemon{{\FA\symbol{"F094}}}
\def\faPhone{{\FA\symbol{"F095}}}
\def\faCheckEmpty{{\FA\symbol{"F096}}}
\def\faBookmarkEmpty{{\FA\symbol{"F097}}}

\newcommand{\faGood}{\textcolor{ForestGreen}{\faThumbsUp}}
\newcommand{\faBad}{\textcolor{red}{\faThumbsODown}}

\usepackage{soul} % for \st that strikes through
\usepackage[normalem]{ulem} % \sout

% \usepackage{minted}
% \newcommand{\inline}[1]{\mintinline{haskell}{#1}}

\title[Part of thesis title]{Функциональщина}
\author{Kakadu}

\institute{St Petersburg University}

\date{\today}

\AtBeginSection[]
{
  \begin{frame}<beamer>
    \frametitle{Outline}
    \tableofcontents[currentsection,currentsubsection]
  \end{frame}
}

\begin{document}
% \maketitle

% For every picture that defines or uses external nodes, you'll have to
% apply the 'remember picture' style. To avoid some typing, we'll apply
% the style to all pictures.
\tikzstyle{every picture}+=[remember picture]

% By default all math in TikZ nodes are set in inline mode. Change this to
% displaystyle so that we don't get small fractions.
\everymath{\displaystyle}

% Uncomment these lines for an automatically generated outline.
% \begin{frame}{Outline}
%   \tableofcontents
% \end{frame}

\section{Objective}

% \begin{frame}{$\Lambda$-исчисление (Church \& Rosser 1936; Church 1941)}
% 
% {\color{teal}\faQuestionSign} Чем отличается \emph{синтаксис} от \emph{семантики}?
% \vspace{1cm}
% 
% Термы -- синтаксические конструкции (здесь -- $\lambda$-исчисления). Термы бывают
% \begin{itemize}
% \item именованные ``переменные'';
% \item ``абстракция'' ($\rightarrow$) строится из имени переменной и терма;
% \item ``применение'' одного терма к другому.
% \end{itemize}
% 
% \pause
% Например,
% \begin{itemize}
%  \item Переменные можно обозначать строчкой букв: $x$, $y$, $z$, и т.д.
%  \item Если терм $B$ удваивает переменную $x$, то терм $\lambda x \rightarrow B$ это терм, обозначающий функцию ``помножить вход на 2''
%  \item Если $f$ терм-функция, которая удваивает вход, и $z$ -- терм-переменная, со семантическим значением 10, то от синтаксической конструкции $f z$ можно ожидать семантическое значение 20.
% \end{itemize}
% 
% \end{frame}

\begin{frame}{Два вида функционального программирования}
На основе Lisp
\begin{itemize}
 \item \emph{Бестиповое}, используем \emph{макросы} чтобы писать было так просто, что нереально допустить ошибку
\end{itemize}
\vspace{0.5cm}
Common Lisp, Scheme, Emacs Lisp, Clojure

\vspace{1cm}\pause
На основе ML
\begin{itemize}
 \item Используем \emph{типы}, чтобы отфильтровать бессмысленные программы
\end{itemize}
\vspace{0.5cm}
Диалекты: Standart ML, OCaml, F\#, \textcolor{ForestGreen}{Haskell}, Scala
\end{frame}

\begin{frame}{Некоторые отличительные собенности...}
\begin{itemize}
 \item Алгебраические типы данных (algebraic datatypes, variant types, discrimintated unions)
 \item Сопоставление с образцом (pattern matching)
 \item Строгая статическая типизация
 \item Легкое  написание DSL используя \emph{комбинаторы}
 \item Автоматический вывод типов
\end{itemize}
\end{frame}

\begin{frame}{...которые перетягивают в известные языки}
С++, C\#, Java, ..., Scala\faQuestion \faExclamation,
\\
\centering{\includegraphics[scale=0.3]{img/lipstick-on-a-pig.jpg}}
\end{frame}




\end{document}
