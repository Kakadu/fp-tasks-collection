\documentclass{beamer}
\usetheme{CambridgeUS}
\beamertemplatenavigationsymbolsempty % remove navigation bar

% \usepackage[utf8]{inputenc}
% \usepackage{default}
% \usepackage[utf8]{inputenc}
% \usepackage[T2A]{fontenc}
%\usepackage{polyglossia}

%\usepackage{fontspec}
%\usepackage[cm-default]{fontspec} % or install lmodern
\usepackage{xltxtra} % load xunicode
\usepackage{polyglossia}
\setmainlanguage{russian}
\setotherlanguage{english}

\setmainfont[
 Ligatures=TeX,
 Extension=.otf,
 BoldFont=cmunbx,
 ItalicFont=cmunti,
 BoldItalicFont=cmunbi,
]{cmunrm}
\setsansfont[
 Ligatures=TeX,
 Extension=.otf,
 BoldFont=cmunsx,
 ItalicFont=cmunsi,
]{cmunss}


\usefonttheme{professionalfonts}
\usepackage{times}
\usepackage{tikz}
\usepackage{amsmath}
\usepackage{verbatim}
\usepackage{graphicx}
\usetikzlibrary{arrows,shapes}


\title[Part of thesis title]{Функциональщина}
\author{Kakadu}

\institute{St Petersburg University}

\date{\today}

\AtBeginSection[]
{
  \begin{frame}<beamer>
    \frametitle{Outline}
    \tableofcontents[currentsection,currentsubsection]
  \end{frame}
}

\begin{document}
\maketitle

\begin{comment}
:Title: Beamer arrows
:Tags: Remember picture, Beamer, Physics & chemistry, Overlays
:Use page: 3

With PGF/TikZ version 1.09 and later, it is possible to draw paths between nodes across
different pictures. This is a useful feature for presentations with the
Beamer package. In this example I've combined the new PGF/TikZ's overlay feature
with Beamer overlays. Download the PDF version to see the result.

**Note.** This only works with PDFTeX, and you have to run PDFTeX twice.

| Author: Kjell Magne Fauske

\end{comment}


% For every picture that defines or uses external nodes, you'll have to
% apply the 'remember picture' style. To avoid some typing, we'll apply
% the style to all pictures.
\tikzstyle{every picture}+=[remember picture]

% By default all math in TikZ nodes are set in inline mode. Change this to
% displaystyle so that we don't get small fractions.
\everymath{\displaystyle}


\begin{frame}
  \titlepage
\end{frame}

% Uncomment these lines for an automatically generated outline.
\begin{frame}{Outline}
  \tableofcontents
\end{frame}


\section{Objective}

\begin{frame}{$\Lambda$ исчисление}
\begin{itemize}
\item Synthesize fiber geometry luminescence solar energy concentrator (LSC) based on polysiloxanes polymer with 22\% and 14\% of diphenyl units in the chain.

\item Avoiding reflection from back signal by black painting and analyzing intensity loss under bended fibers.

\end{itemize}
\end{frame}
% 
% \begin{frame}{$\lambda$ исчисление }
% \begin{itemize}
% \item 1
% 
% \item 2
% \end{itemize}
% \end{frame}
% 
% 
% \begin{frame}{LISP }
% \begin{itemize}
% \item 1
% 
% \item 2
% \end{itemize}
% \end{frame}



\section{Introduction}
\begin{frame}{frame title}
% \begin{itemize}
% \item {LSC is a transparent matrix material such as plastic plate, fiber, coated or doped with organic fluorescent dyes, inorganic phosphors or quantum dots (QD) that absorb sunlight and re-emit at longer wavelengths}
% \item{LSCs is promising  research topic to solve cost effectiveness, efficiency challenges of PV cells and used in buildings due to its flexibility.
% }
% \end{itemize}
\newpage
\begin{itemize}
\item{type of luminescence species and matrix, number and type of solar cell used and also size of LSCs decisively affects the optical efficiency
}
\item{fiber LSC which satisfactorily meets necessary standards of flexibility, by controlling the phenyl functional group in the polysiloxane compound is essential.
}
\item{minimize reflection from back signals, by carbon black painting and light travel along one direction has to be analyzed }
\item{Bending the material results in light intensity loss,the losses relative to radius of bending has to be studied }

\end{itemize}
\end{frame}


% \section{Material and methods}
% \begin{frame}{frame title}
% \end{frame}
% \section{Result and discussion}
% \begin{frame}{frame title}
% % \begin{figure}[h!]
% %    			\includegraphics[scale=0.6]{22.png}
% %             \caption{disk samples}
% % 			\end{figure}
% %             
% % \end{frame}
% \section{Conclusion and future work}
% \begin{frame}{frame title}
% \end{frame}

\end{document}
