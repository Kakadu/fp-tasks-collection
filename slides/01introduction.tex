\documentclass{beamer}
\usetheme{CambridgeUS}
\beamertemplatenavigationsymbolsempty % remove navigation bar

% \usepackage[utf8]{inputenc}
% \usepackage{default}
% \usepackage[utf8]{inputenc}
% \usepackage[T2A]{fontenc}
%\usepackage{polyglossia}

%\usepackage{fontspec}
%\usepackage[cm-default]{fontspec} % or install lmodern
\usepackage{xltxtra} % load xunicode
\usepackage{polyglossia}
\setmainlanguage{russian}
\setotherlanguage{english}
\usepackage{listings}
\lstdefinestyle{myCustomMatlabStyle}{
  language=haskell,
  numbers=left,
  stepnumber=1,
  numbersep=10pt,
  tabsize=4,
  showspaces=false,
  showstringspaces=false
}
\lstset{basicstyle=\large,style=myCustomMatlabStyle}

\setmainfont[
 Ligatures=TeX,
 Extension=.otf,
 BoldFont=cmunbx,
 ItalicFont=cmunti,
 BoldItalicFont=cmunbi,
]{cmunrm}
\setsansfont[
 Ligatures=TeX,
 Extension=.otf,
 BoldFont=cmunsx,
 ItalicFont=cmunsi,
]{cmunss}
% \setmonofont[Scale=0.6]{Monaco}

\usefonttheme{professionalfonts}
\usepackage{times}
\usepackage{tikz}
\usepackage{amsmath}
%\DeclareMathOperator{->}{\rightarrow}
\usepackage{verbatim}
\usepackage{graphicx}
\usetikzlibrary{arrows,shapes}

\usepackage{xcolor}
\definecolor{YellowGreen} {HTML}{B5C28C}
\definecolor{ForestGreen} {HTML}{009B55}

\usepackage{fontspec}

\usepackage{fontawesome}
\expandafter\def\csname faicon@facebook\endcsname{{\FA\symbol{"F09A}}}
\def\faQuestionSign{{\FA\symbol{"F059}}}
\newcommand{\faGood}{\textcolor{ForestGreen}{\faThumbsUp}}
\newcommand{\faBad}{\textcolor{red}{\faThumbsODown}}

% \usepackage{minted}
% \newcommand{\inline}[1]{\mintinline{haskell}{#1}}
\newcommand{\inline}[1]{\lstinline{haskell}{#1}}

\title[Part of thesis title]{Функциональщина}
\author{Kakadu}

\institute{St Petersburg University}

\date{\today}

\AtBeginSection[]
{
  \begin{frame}<beamer>
    \frametitle{Outline}
    \tableofcontents[currentsection,currentsubsection]
  \end{frame}
}

\begin{document}
\maketitle

\begin{comment}
:Title: Beamer arrows
:Tags: Remember picture, Beamer, Physics & chemistry, Overlays
:Use page: 3

With PGF/TikZ version 1.09 and later, it is possible to draw paths between nodes across
different pictures. This is a useful feature for presentations with the
Beamer package. In this example I've combined the new PGF/TikZ's overlay feature
with Beamer overlays. Download the PDF version to see the result.

**Note.** This only works with PDFTeX, and you have to run PDFTeX twice.

| Author: Kjell Magne Fauske

\end{comment}


% For every picture that defines or uses external nodes, you'll have to
% apply the 'remember picture' style. To avoid some typing, we'll apply
% the style to all pictures.
\tikzstyle{every picture}+=[remember picture]

% By default all math in TikZ nodes are set in inline mode. Change this to
% displaystyle so that we don't get small fractions.
\everymath{\displaystyle}


\begin{frame}
  \titlepage
\end{frame}

% Uncomment these lines for an automatically generated outline.
\begin{frame}{Outline}
  \tableofcontents
\end{frame}


\section{Objective}

% \begin{frame}{$\Lambda$-исчисление (Church \& Rosser 1936; Church 1941)}
% 
% {\color{teal}\faQuestionSign} Чем отличается \emph{синтаксис} от \emph{семантики}?
% \vspace{1cm}
% 
% Термы -- синтаксические конструкции (здесь -- $\lambda$-исчисления). Термы бывают
% \begin{itemize}
% \item именованные ``переменные'';
% \item ``абстракция'' ($\rightarrow$) строится из имени переменной и терма;
% \item ``применение'' одного терма к другому.
% \end{itemize}
% 
% \pause
% Например,
% \begin{itemize}
%  \item Переменные можно обозначать строчкой букв: $x$, $y$, $z$, и т.д.
%  \item Если терм $B$ удваивает переменную $x$, то терм $\lambda x \rightarrow B$ это терм, обозначающий функцию ``помножить вход на 2''
%  \item Если $f$ терм-функция, которая удваивает вход, и $z$ -- терм-переменная, со семантическим значением 10, то от синтаксической конструкции $f z$ можно ожидать семантическое значение 20.
% \end{itemize}
% 
% \end{frame}

\begin{frame}{Два вида функционального программирования}
На основе Lisp
\begin{itemize}
 \item \emph{Бестиповое}, используем \emph{макросы} чтобы писать было так просто, что нереально допустить ошибку
\end{itemize}
\vspace{0.5cm}
Common Lisp, Scheme, Emacs Lisp, Clojure

\vspace{1cm}\pause
На основе ML
\begin{itemize}
 \item Используем \emph{типы}, чтобы отфильтровать бессмысленные программы
\end{itemize}
\vspace{0.5cm}
Диалекты: Standart ML, OCaml, F\#, \textcolor{ForestGreen}{Haskell}, Scala

\end{frame}

\begin{frame}{Особенности}
\begin{itemize}
 \item Алгебраические типы данных (algebraic datatypes, variant types, discrimintated unions)
 \item Сопоставление с образцом (pattern matching)
 \item Строгая статическая типизация
 \item Легкое  написание DSL используя \emph{комбинаторы}
\end{itemize}
\end{frame}


\section{Типы}

\begin{frame}[fragile]{Синтаксис типов в Haskell (1/2)}
% ~\ \emph{Тип функции}, действующей из аргумента типа $\mathcal{A}$ и возвращающей результат $\mathcal{B}$, обозначается как \lstinline{data X = A -> B}.
% % \\ \pause
% ~\ Функции, у которых $n$ аргументов ($n>1$) моделируются как функции возвращающие функцию от $n-1$ аргументов. Например, $\mathcal{-> (B\rightarrow C)}$. Ассоциативность правая: $\mathcal{-> (B\rightarrow C)} = \mathcal{-> B\rightarrow C}$
 % 
\begin{lstlisting} [basicstyle=\ttfamily]
Put your code here.1iI
\end{lstlisting}

\end{frame}

\begin{frame}{Синтаксис типов в Haskell (2/2)}
~\ \emph{Типовые переменные} в типах полиморфных функций пишутся с маленькой буквы. Например, $a -> b$ или $a -> b -> c$ или $a -> (a -> b) -> b$ 
\\ \pause
~\ \emph{Имена типов} пишутся с заглавной буквы. Например,  \lstinline=Int=, \lstinline=String=, \lstinline=Float=.
\\ \pause
~\ \emph{Типы с параметрами}: \lstinline=List Int=, \lstinline=List String=, \lstinline=List a=, и т.д. 
\end{frame}



% \section{Material and methods}
% \begin{frame}{frame title}
% \end{frame}
% \section{Result and discussion}
% \begin{frame}{frame title}
% % \begin{figure}[h!]
% %    			\includegraphics[scale=0.6]{22.png}
% %             \caption{disk samples}
% % 			\end{figure}
% %             
% % \end{frame}
% \section{Conclusion and future work}
% \begin{frame}{frame title}
% \end{frame}

\end{document}
