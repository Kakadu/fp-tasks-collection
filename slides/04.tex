\documentclass{beamer}
\usetheme{CambridgeUS}
\beamertemplatenavigationsymbolsempty % remove navigation bar

\input{heading2.tex}


\title[Занятие 4: типы]{Типы в функциональном программировании}
\author{Косарев Дмитрий a.k.a. Kakadu}

\institute{матмех СПбГУ}

\date{\today}

\AtBeginSection[]
{
  \begin{frame}<beamer>
    \frametitle{Outline}
    \tableofcontents[currentsection,currentsubsection]
  \end{frame}
}

\begin{document}
\maketitle

% For every picture that defines or uses external nodes, you'll have to
% apply the 'remember picture' style. To avoid some typing, we'll apply
% the style to all pictures.
\tikzstyle{every picture}+=[remember picture]

% By default all math in TikZ nodes are set in inline mode. Change this to
% displaystyle so that we don't get small fractions.
\everymath{\displaystyle}

% Uncomment these lines for an automatically generated outline.
% \begin{frame}{Outline}
%   \tableofcontents
% \end{frame}

\begin{frame}[fragile]{Какие-то задачки}
\pause
~\ Ну, если $x$ принадлежит типу \inline=T=, то \inline=T= определяет, какие значения может принимать $x$.
\\ \vspace{0.5cm} %\pause
~\ \emph{Тип} \inline=T= у $x$ -- это \sout{множество} совокупность значений, которые могут быть у $x$.
\vspace{0.5cm} \\ %\pause
~\  Если \inline=x :: T=, то тип \inline=T= \emph{населен} иксом.
\\ \vspace{0.5cm} %\pause
~\ Если $\nexists$ \inline=x=, таких что \inline=x :: T=, то тип \inline=T= \emph{не населен}.
\end{frame}


\end{document}
