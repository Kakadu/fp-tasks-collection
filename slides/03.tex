\documentclass[
  xcolor={svgnames},
  hyperref={colorlinks,citecolor=DeepPink4,linkcolor=DarkRed,urlcolor=DarkBlue}]{beamer}
\usetheme{CambridgeUS}
\beamertemplatenavigationsymbolsempty % remove navigation bar

\input{heading2.tex}

\title[Занятие 3]{Несколько слов о несвязных темах}
\author{Косарев Дмитрий a.k.a. Kakadu}

\institute{матмех СПбГУ}

\date{\today}

\AtBeginSection[]
{
  \begin{frame}<beamer>
    \frametitle{Outline}
    \tableofcontents[currentsection,currentsubsection]
  \end{frame}
}

\newcommand{\verbatimfont}[1]{\def\verbatim@font{#1}}
\usepackage{verbatimbox}
\begin{document}
\maketitle

% For every picture that defines or uses external nodes, you'll have to
% apply the 'remember picture' style. To avoid some typing, we'll apply
% the style to all pictures.
\tikzstyle{every picture}+=[remember picture]

% By default all math in TikZ nodes are set in inline mode. Change this to
% displaystyle so that we don't get small fractions.
\everymath{\displaystyle}

% Uncomment these lines for an automatically generated outline.
% \begin{frame}{Outline}
%   \tableofcontents
% \end{frame}

\begin{frame}[fragile]{Несколько задачек про прокрастинирующие вычисления}
\begin{itemize}
 \item Построить бесконечный список чисел Фибоначчи \\ \href{https://www.haskell.org/hoogle/?hoogle=zipWith}{Подсказка}:\vspace{0.5cm}
 \begin{verbnobox}[\monacoB]
zipWith :: (a -> b -> c) -> [a] -> [b] -> [c]
 \end{verbnobox}
 \item Построить список простых чисел (решето Эратосфена)
 \item Дан (бесконечный) список (бесконечных) списков. Сложить все элементы в один большой список. Ограничение: если элемент стоит в списке с конечным номером на позиции с конечным номером, то в результируюещем списке он должен быть наблюдаем на конечной позиции.
\end{itemize}

\end{frame}

\begin{frame}[fragile]{Жесткая задача про прокрастинирующие вычисления}
Есть бинарное дерево со значениями только в листях и типом \pause
\begin{verbnobox}[\monacoB] 
data Tree a = Leaf a
            | Node (Tree a) (Tree a)
\end{verbnobox}
Дано дерево типа \hsinline=Tree Int=. Нужно построить новое дерево такой же структуры, складывая в листья минимальный элемент исходного дерева
\begin{itemize}
 \item за 2 прохода
 \item за 1 проход
\end{itemize}
\end{frame}

\begin{frame}[fragile]{Логика высказываний}
Формулы бывают:
\begin{itemize}
 \item Переменные: $a$, $b$, $c$ ...
 \item Две логические константы
 \item Отрицания формул:  $\neg a$, $\neg\neg b$ ...
 \item Конъюнкции двух формул: $a \wedge b$, $\neg b \wedge c$ ...
 \item Дизъюнкции двух формул: $a \vee b$, $\neg b \vee (c \vee d)$ ...
 \item Имплицации двух формул: $a \supset b$, $\neg b \supset (c \supset d)$ ...
\end{itemize}
\pause
Можно \textit{интерпретировать} синтаксические конструкции выше, придав смысл переменным, отрицанию и бинарным операциям. Google: логика высказываний, булева алгебра, таблица истинности.
\end{frame}

\begin{frame}[fragile]{Упражнения про логику высказываний}
\begin{itemize}
 \item Опишите тип данных \hsinline=data Formula = ...=
 \item Дана формула и окружение, которое связывает переменные. Упростите формулу пока упрощается
 \item Построить ДНФ по произвольной формуле
 \item Построить КНФ по произвольной формуле
 \item Дана формула с переменными. Проверить, что она истинная для любых значений переменных
  \begin{itemize}
   \item Можно решать перебором
   \item Можно придумать что-то хитрое
  \end{itemize}
\end{itemize}
\end{frame}

\end{document}
