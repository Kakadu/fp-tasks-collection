\documentclass[
  xcolor={svgnames},
  hyperref={colorlinks,citecolor=DeepPink4,linkcolor=DarkRed,urlcolor=DarkBlue}]{beamer}
\usetheme{CambridgeUS}
\beamertemplatenavigationsymbolsempty % remove navigation bar

\usepackage{xcolor}
\definecolor{YellowGreen} {HTML}{B5C28C}
\definecolor{ForestGreen} {HTML}{009B55}
\definecolor{MyBackground}{HTML}{F0EDAA}


\usepackage{xltxtra} % load xunicode
\usepackage{polyglossia}
\setmainlanguage{russian}
\setotherlanguage{english}

\let\cyrillicfonttt\monofamily
\newfontfamily\dejaVuSansMono{DejaVu Sans Mono}
% https://github.com/vjpr/monaco-bold/raw/master/MonacoB/MonacoB.otf
\newfontfamily\monacoB{MonacoB}

\usepackage{listings}
\lstdefinestyle{style1}{
  language=haskell,
  numbers=left,
  stepnumber=1,
  numbersep=10pt,
  tabsize=4,
  showspaces=false,
  showstringspaces=false
}
\lstdefinestyle{hsstyle1}
{ language=haskell
         , basicstyle=\monacoB
         , deletekeywords={Int,Float,String,List,Void}
         , breaklines=true
         , columns=fullflexible
         , commentstyle=\color{ForestGreen}
         , escapeinside=§§
         , escapebegin=\begin{russian}\commentfont
         , escapeend=\end{russian}
         , commentstyle=\color{ForestGreen}
         , escapeinside=§§
         , escapebegin=\begin{russian}\monacoB\color{ForestGreen}
         , escapeend=\end{russian}
         , mathescape=true
%          , backgroundcolor = \color{MyBackground}
}

\newcommand{\inline}[1]{\lstinline{haskell}{#1}}
\def\hsinline{\lstinline[style={hsstyle1}]}
\def\inline{\hsinline}

\lstnewenvironment{hslisting} {
    \lstset { style={hsstyle1} }
  }
  {}
  
%%%%%%%%%%%%%%%%%%%%%%%%%%%%%%%%%%%%%%%%%%%%%%%%%%%%%%%%%%  
\setmainfont[
 Ligatures=TeX,
 Extension=.otf,
 BoldFont=cmunbx,
 ItalicFont=cmunti,
 BoldItalicFont=cmunbi,
]{cmunrm}
% С засечками (для заголовков)
\setsansfont[
 Ligatures=TeX,
 Extension=.otf,
 BoldFont=cmunsx,
 ItalicFont=cmunsi,
]{cmunss}
% \setmonofont[Scale=0.6]{Monaco}

\usefonttheme{professionalfonts}
\usepackage{times}
\usepackage{tikz}
\usetikzlibrary{cd}
% \usepackage{tikz-cd}
\usepackage{amsmath}
%\DeclareMathOperator{->}{\rightarrow}
\newcommand\iso{\ensuremath{\cong}}
\usepackage{verbatim}
\usepackage{graphicx}
\usetikzlibrary{arrows,shapes}

\usepackage{fontawesome}
% \newfontfamily{\FA}{Font Awesome 5 Free} % some glyphs missing
\expandafter\def\csname faicon@facebook\endcsname{{\FA\symbol{"F09A}}}
\def\faQuestionSign{{\FA\symbol{"F059}}}
\def\faQuestion{{\FA\symbol{"F128}}}
\def\faExclamation{{\FA\symbol{"F12A}}}
\def\faUploadAlt{{\FA\symbol{"F093}}}
\def\faLemon{{\FA\symbol{"F094}}}
\def\faPhone{{\FA\symbol{"F095}}}
\def\faCheckEmpty{{\FA\symbol{"F096}}}
\def\faBookmarkEmpty{{\FA\symbol{"F097}}}

\newcommand{\faGood}{\textcolor{ForestGreen}{\faThumbsUp}}
\newcommand{\faBad}{\textcolor{red}{\faThumbsODown}}

\usepackage{soul} % for \st that strikes through
\usepackage[normalem]{ulem} % \sout

\title[Занятие 3]{Несколько задая. Стратегии вычислений. Fix}
\author{Косарев Дмитрий a.k.a. Kakadu}

\institute{матмех СПбГУ}

\date{\today}

\AtBeginSection[]
{
  \begin{frame}<beamer>
    \frametitle{Outline}
    \tableofcontents[currentsection,currentsubsection]
  \end{frame}
}

\newcommand{\verbatimfont}[1]{\def\verbatim@font{#1}}
\usepackage{verbatimbox}
\begin{document}
\maketitle

% For every picture that defines or uses external nodes, you'll have to
% apply the 'remember picture' style. To avoid some typing, we'll apply
% the style to all pictures.
\tikzstyle{every picture}+=[remember picture]

% By default all math in TikZ nodes are set in inline mode. Change this to
% displaystyle so that we don't get small fractions.
\everymath{\displaystyle}

% Uncomment these lines for an automatically generated outline.
% \begin{frame}{Outline}
%   \tableofcontents
% \end{frame}

\section{Несколько задач}

\begin{frame}[fragile]{Несколько задачек про прокрастинирующие вычисления}
\begin{itemize}
 \item Построить бесконечный список чисел Фибоначчи \\ \href{https://www.haskell.org/hoogle/?hoogle=zipWith}{Подсказка}:\vspace{0.5cm}
 \begin{verbnobox}[\monacoB]
zipWith :: (a -> b -> c) -> [a] -> [b] -> [c]
 \end{verbnobox}
 \item Построить список простых чисел (решето Эратосфена)
 \item Дан (бесконечный) список (бесконечных) списков. Сложить все элементы в один большой список. Ограничение: если элемент стоит в списке с конечным номером на позиции с конечным номером, то в результируюещем списке он должен быть наблюдаем на конечной позиции.
\end{itemize}

\end{frame}

\begin{frame}[fragile]{Жесткая задача про прокрастинирующие вычисления}
Есть бинарное дерево со значениями только в листях и типом \pause
\begin{verbnobox}[\monacoB] 
data Tree a = Leaf a
            | Node (Tree a) (Tree a)
\end{verbnobox}
Дано дерево типа \hsinline=Tree Int=. Нужно построить новое дерево такой же структуры, складывая в листья минимальный элемент исходного дерева
\begin{itemize}
 \item за 2 прохода
 \item за 1 проход
\end{itemize}
\end{frame}

\begin{frame}[fragile]{Логика высказываний}
Формулы бывают:
\begin{itemize}
 \item Переменные: $a$, $b$, $c$ ...
 \item Две логические константы
 \item Отрицания формул:  $\neg a$, $\neg\neg b$ ...
 \item Конъюнкции двух формул: $a \wedge b$, $\neg b \wedge c$ ...
 \item Дизъюнкции двух формул: $a \vee b$, $\neg b \vee (c \vee d)$ ...
 \item Импликации двух формул: $a \supset b$, $\neg b \supset (c \supset d)$ ...
\end{itemize}
\pause
Можно \textit{интерпретировать} синтаксические конструкции выше, придав смысл переменным, отрицанию и бинарным операциям. Google: логика высказываний, булева алгебра, таблица истинности.
\end{frame}

\begin{frame}[fragile]{Упражнения про логику высказываний}
\begin{itemize}
 \item Опишите тип данных \hsinline=data Formula = ...=
 \item Дана формула и окружение, которое связывает переменные. Упростите формулу пока упрощается
 \item Построить ДНФ по произвольной формуле
 \item Построить КНФ по произвольной формуле
 \item Дана формула с переменными. Проверить, что она истинная для любых значений переменных
  \begin{itemize}
   \item Можно решать перебором
   \item Можно придумать что-то хитрое
  \end{itemize}
\end{itemize}
\end{frame}

\section{Мини-язык $\Lambda$ с арифметикой}

\begin{frame}[fragile]{Ещё один мини-язык $\Lambda$ с арифметикой}
Выражения в мини-языке бывают:
\begin{itemize}
 \item Константы: $1$,$2$,$3$, и т.д.
 \item Именованные значения (``переменные''): $a$,$b$,$x$,$y$, и т.д.
 \item Оператор сложения \verb=+=: $a+1$, $x+y+x$, и т.д.
 \item Функции от одного аргумента: $(\lambda x \rightarrow x+1)$, $(\lambda x \rightarrow \lambda y \rightarrow x+y)$, и т.д.
 \item Применение одного выражения к другому: $(\lambda x \rightarrow x + x) (2+3)$, $((\lambda x \rightarrow \lambda y \rightarrow x + y) 2) (3+5)$, и т.д.
\end{itemize}
\pause
Упражнение: напишите алгебраический тип для языка $\Lambda$.\\
\pause
{\Large \faQuestion} Если мы хотим написать интерпретатор $\Lambda$, то как он должен работать?
\end{frame}

\begin{frame}[fragile]{Call-by-value}
Также известно как \textit{stict} или \textit{eager}. Применяется в большинстве известных языков программирования.
\vspace{1cm}

Описывает как вычислять применение \hsinline=a= к \hsinline=b=.
\begin{itemize}
 \item Слева направо: вычисляем \hsinline=a=, затем \hsinline=b=, затем применяем.
 \item Справа налево: \hsinline=b=, \hsinline=a=, затем применяем.
\end{itemize}

Замечание: если \hsinline=a= посчитано до конца, то разница отсутствует.
\end{frame}

\begin{frame}[fragile]{Call-by-value: примеры}
Слева направо:
\begin{itemize}
 \item $((\lambda x \rightarrow \lambda y \rightarrow x + y) 2) (3+5)$
 \item $(\lambda y \rightarrow 2 + y) (3+5)$
 \item $(\lambda y \rightarrow 2 + y) 8$
 \item $2+8$
 \item $10$
\end{itemize}
Справа налево:
\begin{itemize}
 \item $((\lambda x \rightarrow \lambda y \rightarrow x + y) 2) (3+5)$
 \item $((\lambda x \rightarrow \lambda y \rightarrow x + y) 2) 8$
 \item $(\lambda y \rightarrow 2 + y) 8$
 \item $2+8$
 \item $10$
\end{itemize}
В чистых языках не важно как вычислять: справо налево или слева направо, ответ одинаковый.
\end{frame}

\begin{frame}[fragile]{Call-by-name: пример}
Не вычисляем аргменты функций. Когда они понадобятся мы подставляем аргумент как он есть туда. где он используется.
\vspace{1cm}
\begin{itemize}
 \item $((\lambda x \rightarrow \lambda y \rightarrow x + y + y) 2) (3+5)$
 \item $(\lambda y \rightarrow 2 + y + y) (3+5)$
 \item $2 + (3+5) + (3+5)$
 \item $2 + 8 + (3+5)$ 
 \item А теперь допустим, что \verb=+= вычисляется слева направо
 \item $10 + (3+5)$
 \item $10+8$
 \item $18$
\end{itemize}
\end{frame}

\begin{frame}[fragile]{Call-by-name: пример и не из мини-языка, и не из Haskell}
\begin{itemize}
 \item $(\lambda x \rightarrow x; x)$ (print "hi")
 \item (print "hi"); (print "hi")
 \item (print "hi")
 \item ()
\end{itemize}
{\Large \faQuestion} Что напечатается?
\end{frame}

\begin{frame}[fragile]{Вычисления ... особого рода}
Упражнение прямо на лекции: давайте посчитаем выражение тремя способами\\
\begin{center}
 $(\lambda y \rightarrow y y) (\lambda x \rightarrow x x)$
\end{center}
\pause
Считаем, считаем, считаем... и получаем
\begin{center}
 $(\lambda x \rightarrow x x) (\lambda x \rightarrow x x)$
\end{center}
То же самое, с точностью до переименования\\
Считаем дальше и зависаем\\
\vspace{1cm}
Одно и то же, при трех стратегиях.
\end{frame}

\begin{frame}[fragile]{Всегда ли одинаковый ответ?}
Положим $loop$ это $(\lambda x \rightarrow x x) (\lambda x \rightarrow x x)$ и рассмотрим
\begin{center}
 $(\lambda x \rightarrow\lambda y \rightarrow y y) (loop)$
\end{center}\pause
{\Large \faQuestion} Что будет, если посчитать с помощью CBV?\pause \\
{\Large \faQuestion} А если CBN?\pause
\begin{center}
 $(\lambda y \rightarrow y)$
\end{center}
CBN завершается, а CBV -- нет
\end{frame}

\begin{frame}[fragile]{Так может CBV строго лучше CBN?}
\begin{center}
 $(\lambda y \rightarrow y y) (big)$
\end{center}
Применяем call-by-name:\pause
\begin{center}
 $(big)(big)$
\end{center}\pause
CBV считает аргумент один раз. CBN -- два. \\
\vspace{1cm}
Вообще, если аргумент функции встречается больше одного раза, то CBN будет вычислять его больше одного раза. Нехорошо.
\end{frame}

\begin{frame}[fragile]{Так может CBV vs. CBN vs. call-by-need (lazy, прокрастинирующий)}
Иногда CBN завершается, а CBV -- нет.\\
Иногда CBV не вычисляет что-то, что CBN вычисляет.\\
Иногда CBN не вычисляет что-то 2,3 или больше раз, когда CBV только один.\\
\vspace{1cm}
Ленивые вычисления
\begin{itemize}
 \item Если что-то посчитали -- запоминаем результат на будущее. Избегает повторных вычислений
 \item Завершается, если CBN завершается
 \item Вычисляет аргументы максимум один раз
\end{itemize}
{\Large \faQuestion} Так может call-by-need -- это то, что нужно всегда использовать?
\end{frame}

\begin{frame}[fragile]{call-by-need a.k.a. прокрастинирующие}
\begin{itemize}
  \item Создание отложенного вычисления типа \hsinline=T= зачастую сводится к созданию функции \hsinline=() -> T=.
  \item Требует некоторых усилий, использует некоторое количество памяти. На больших программах может стать важным
\end{itemize}
\begin{verbnobox}[\monacoB]
 xs = [1,2,3,..... big list, ...]
 n = foldl (+) 0 xs
\end{verbnobox}
\pause
Список \hsinline=xs= занимает много памяти, но его нельзя объявить мусором, потому что элементы используются при вычислении \hsinline=n=.\pause\\
\vspace{0.5cm}
Но существует функция \hsinline=fold'=!\\

Упражнение: попробуйте поскладывать большие списки чисел (\hsinline=[1..1000000]=) с помощью foldl, foldr, foldl'. Когда лучше использовать \hsinline=foldl= вместо \hsinline=foldl'=?
\end{frame}

\begin{frame}[fragile]{Упражнение: вычисление выражения из языка $\Lambda$}
Задачи такого рода были в том году на экзамене.
\vspace{1cm}
\hsinline=evalL:: [(String, L)] -> L -> Maybe L=

Нужно действовать по аналогии с примером про вычислитель арифметики с операциями сложения и умножения. Из стратегий вычислений (CBV, CBN, прокрастинирующую) выберите какая нравится.

N.B. Там есть \textbf{грабли}.
\end{frame}

\section{Про Fix}
\begin{frame}[fragile]{Неподвижная точка}
Рассмотрим $f: A \rightarrow A$. \\
$x$ -- это \textit{неподвижная точка}, если $f(x) = x$.

\vspace{1cm}
Неподвижная точка в Haskell:
\begin{verbnobox}[\monacoB]
 fix f = let {x = f x} in x
\end{verbnobox}
ну или
\begin{verbnobox}[\monacoB]
 fix f = f (fix f)
\end{verbnobox}
\end{frame}

\begin{frame}[fragile]{Пример}
\begin{verbnobox}[\monacoB]
fix :: (t -> t) -> t
fix f = f (fix f)
\end{verbnobox}
 
\begin{verbnobox}[\monacoB]
last :: [a] -> Maybe a
last [] = Nothing
last [x] = Just x
last (x:xs) = last xs
\end{verbnobox}
\pause
\begin{verbnobox}[\monacoB]
last2 :: [a] -> Maybe a
last2 = fix g 
  where 
    g _ []     = Nothing
    g _ [x]    = Just x
    g r (_:xs) = r xs
\end{verbnobox}
\end{frame}

\begin{frame}[fragile]{Упражнения про fix для функций}
 Реализовать через \hsinline=fix=:
 \begin{itemize}
  \item \hsinline=map=
  \item Вычисление n-го числа Фибоначчи
  \item Для данного числа \hsinline=n= посчитайте \hsinline=cos (cos (cos ..... n)...)= пока результат (а именно \hsinline=0.7390851332151607=) не перестанет изменяться. 
 \end{itemize}

\end{frame}

\begin{frame}[fragile]{Fixpoint для типов (1/2) }
Почти связный список
\begin{center}
\hsinline!data ListG a r = Nil | Cons a r!
\end{center}
Что по сути описывает либо пустой список, либо ячейку памяти со значением типа \hsinline=a= и ``дыркой''.

\begin{center}
\hsinline!type List2 a r = ListG a (ListG a r)!
\hsinline!type List3 a r = ListG a (List2 a r)!
\hsinline!type List4 a r = ListG a (List3 a r)!
\end{center}

\end{frame}

\begin{frame}[fragile]{Fixpoint для типов (2/2)}
 
\begin{lstlisting}[style={hsstyle1}, mathescape=true, escapebegin=, escapeend=]
type RingF$_{n+1}$ a = RingF   (RingF$_{n+1}$  a)

newtype Fix f = Fix (f (Fix f))
Fix :: f (Fix f) -> Fix f

type List a = Fix (ListG a)
\end{lstlisting}
Теперь можно конструировать значения типа \hsinline=List a=:\pause
\begin{lstlisting}[style={hsstyle1}, mathescape=false, escapebegin=, escapeend=]
(Fix Nil) :: List a
(Fix $ Cons 1  $ Fix Nil) :: List Int
(Fix $ Cons 1  $ Fix $ Cons 2 $ Fix Nil) :: List Int :: List Int
\end{lstlisting}
\end{frame}

\end{document}
