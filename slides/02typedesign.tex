\documentclass{beamer}
\usetheme{CambridgeUS}
\beamertemplatenavigationsymbolsempty % remove navigation bar

\usepackage{xcolor}
\definecolor{YellowGreen} {HTML}{B5C28C}
\definecolor{ForestGreen} {HTML}{009B55}
\definecolor{MyBackground}{HTML}{F0EDAA}


\usepackage{xltxtra} % load xunicode
\usepackage{polyglossia}
\setmainlanguage{russian}
\setotherlanguage{english}

\let\cyrillicfonttt\monofamily
\newfontfamily\dejaVuSansMono{DejaVu Sans Mono}
% https://github.com/vjpr/monaco-bold/raw/master/MonacoB/MonacoB.otf
\newfontfamily\monacoB{MonacoB}

\usepackage{listings}
\lstdefinestyle{style1}{
  language=haskell,
  numbers=left,
  stepnumber=1,
  numbersep=10pt,
  tabsize=4,
  showspaces=false,
  showstringspaces=false
}
\lstdefinestyle{hsstyle1}
{ language=haskell
         , basicstyle=\monacoB
         , deletekeywords={Int,Float,String,List,Void}
         , breaklines=true
         , columns=fullflexible
         , commentstyle=\color{ForestGreen}
         , escapeinside=§§
         , escapebegin=\begin{russian}\commentfont
         , escapeend=\end{russian}
         , commentstyle=\color{ForestGreen}
         , escapeinside=§§
         , escapebegin=\begin{russian}\monacoB\color{ForestGreen}
         , escapeend=\end{russian}
         , mathescape=true
%          , backgroundcolor = \color{MyBackground}
}

\newcommand{\inline}[1]{\lstinline{haskell}{#1}}
\def\hsinline{\lstinline[style={hsstyle1}]}
\def\inline{\hsinline}

\lstnewenvironment{hslisting} {
    \lstset { style={hsstyle1} }
  }
  {}
  
%%%%%%%%%%%%%%%%%%%%%%%%%%%%%%%%%%%%%%%%%%%%%%%%%%%%%%%%%%  
\setmainfont[
 Ligatures=TeX,
 Extension=.otf,
 BoldFont=cmunbx,
 ItalicFont=cmunti,
 BoldItalicFont=cmunbi,
]{cmunrm}
% С засечками (для заголовков)
\setsansfont[
 Ligatures=TeX,
 Extension=.otf,
 BoldFont=cmunsx,
 ItalicFont=cmunsi,
]{cmunss}
% \setmonofont[Scale=0.6]{Monaco}

\usefonttheme{professionalfonts}
\usepackage{times}
\usepackage{tikz}
\usetikzlibrary{cd}
% \usepackage{tikz-cd}
\usepackage{amsmath}
%\DeclareMathOperator{->}{\rightarrow}
\newcommand\iso{\ensuremath{\cong}}
\usepackage{verbatim}
\usepackage{graphicx}
\usetikzlibrary{arrows,shapes}

\usepackage{fontawesome}
% \newfontfamily{\FA}{Font Awesome 5 Free} % some glyphs missing
\expandafter\def\csname faicon@facebook\endcsname{{\FA\symbol{"F09A}}}
\def\faQuestionSign{{\FA\symbol{"F059}}}
\def\faQuestion{{\FA\symbol{"F128}}}
\def\faExclamation{{\FA\symbol{"F12A}}}
\def\faUploadAlt{{\FA\symbol{"F093}}}
\def\faLemon{{\FA\symbol{"F094}}}
\def\faPhone{{\FA\symbol{"F095}}}
\def\faCheckEmpty{{\FA\symbol{"F096}}}
\def\faBookmarkEmpty{{\FA\symbol{"F097}}}

\newcommand{\faGood}{\textcolor{ForestGreen}{\faThumbsUp}}
\newcommand{\faBad}{\textcolor{red}{\faThumbsODown}}

\usepackage{soul} % for \st that strikes through
\usepackage[normalem]{ulem} % \sout

\title[Занятие 2]{Пользовательские типы дынных}
\author{Косарев Дмитрий a.k.a. Kakadu}

\institute{матмех СПбГУ}

\date{\today}

\AtBeginSection[]
{
  \begin{frame}<beamer>
    \frametitle{Outline}
    \tableofcontents[currentsection,currentsubsection]
  \end{frame}
}

\begin{document}
\maketitle

% For every picture that defines or uses external nodes, you'll have to
% apply the 'remember picture' style. To avoid some typing, we'll apply
% the style to all pictures.
\tikzstyle{every picture}+=[remember picture]

% By default all math in TikZ nodes are set in inline mode. Change this to
% displaystyle so that we don't get small fractions.
\everymath{\displaystyle}

% Uncomment these lines for an automatically generated outline.
% \begin{frame}{Outline}
%   \tableofcontents
% \end{frame}

\begin{frame}[fragile]{Синтаксис типов в Haskell (1/2)}
\begin{hslisting}
data InetAddr = InetAddr Word8 Word8 Word8 Word8

data ConnectionState = 
  Connecting  | Connected  | Disconnected 
  
data ConnectionInfo = CInfo 
  { state ::                   ConnectionState
  , server ::                  InetAddr
  , last_ping_time ::          Maybe Time
  , last_ping_id ::            Maybe Int
  , session_id ::              Maybe String
  , when_initiated ::          Maybe Time
  , when_disconnected ::       Maybe Time
  } 
\end{hslisting}
\end{frame}

\begin{frame}[fragile]{?}
\begin{hslisting}
data Connecting = 
  Connecting { when_initiated:: Time } 
data Connected  = Connected 
  { last_ping  :: Maybe (Time,Int)
  , session_id :: String   } 
data Disconnected = 
  Disconnected { when_disconnected:: Time } 
 
data ConnectionState = 
  | SConnecting   Connecting 
  | SConnected    Connected 
  | SDisconnected Disconnected 
\end{hslisting}
\end{frame}

\begin{frame}[fragile]{?}
\begin{hslisting}
data ConnectionState = 
  | SConnecting   Connecting 
  | SConnected    Connected 
  | SDisconnected Disconnected 
 
data ConnectionInfo = Cinfo 
  { state ::  ConnectionState
  , server:: InetAddr } 
\end{hslisting}
\end{frame}

\begin{frame}[fragile]{Простой пример: арифметика}
\begin{hslisting}
data Expr = 
  | EConst Int
  | EMul Expr Expr
  | EAdd Expr Expr
 
eval :: Expr -> Int
\end{hslisting}
\pause
\begin{hslisting}
eval (EConst n) = n
eval (EAdd l r) = (eval l) + (eval r)
eval (EMul l r) = (eval l) * (eval r)
\end{hslisting}

\end{frame}

\begin{frame}[fragile]{Арифметика с переменными}
\begin{hslisting}
data Expr = 
  | EConst Int
  | EMul Expr Expr
  | EAdd Expr Expr
  | EVar String
 
eval :: [(String,Int)] -> Expr -> Maybe Int
\end{hslisting}
\pause
\begin{hslisting}
eval _ (EConst n) = Just n
eval env (EAdd l r) = 
  Just ((eval env l) + (eval env r))
eval env (EMul l r) = 
  Just ((eval env l) * (eval env r))
eval env (EVar s) = lookup s env 
\end{hslisting}

\end{frame}

% \begin{frame}{Синтаксис типов в Haskell (2/2)}
% ~\ \emph{Параметрический полиморфизм} -- когда один и тот же код, работает для 
% разных типов (generics в Java/C\#).
% \\ \pause
% ~\ Ещё бывает \emph{ad-hoc полиморфизм} -- работает разный код для разных типов (overloading в С++). Но это потом.
% \\ \pause
% ~\ \emph{Типовые переменные} в типах полиморфных функций пишутся с маленькой буквы. Например, \inline{a -> b} или \inline{a -> b -> c} или 
% \hsinline=a -> (a -> b) -> b=
% \vspace{0.5cm} 
% \\ \pause
% ~\ \emph{Имена типов} пишутся с заглавной буквы. Например,  \inline=Int=, \inline=String=, \inline=Float=.
% \vspace{0.5cm} 
% \\ \pause
% ~\ Когда какое-то имя \emph{типизируется как}(\emph{имеет тип}) \inline=T=, то это записывают так: \inline=x :: T=.
% \end{frame}
% 
% \begin{frame}{Применение типов}
% ~\ \emph{Применение типов(type application)}: \inline=List Int=, \inline=List String=, \inline=List a=, и т.д. \\
% \vspace{1cm}
% (Тут должна быть картинка)
%  
% \end{frame}
% 
% \begin{frame}{Чем функции в программировании отличаются от математических?}
%  \pause
%  \begin{itemize}
%   \item Аварийное завершение
%   \item Отсутствие завершения
%  \end{itemize}
% \end{frame}
% 
% \begin{frame}[fragile]{Примеры параметрического полиморфизма}
% \begin{hslisting}
% Prelude> let id x = x
% Prelude> :t id
% id :: p -> p
% \end{hslisting}
% \pause
% \begin{hslisting}
% Prelude> :t (id :: String -> String)
% (id :: String -> String) :: String -> String
% \end{hslisting}
% \pause
% \begin{hslisting}
% Prelude> :t (id :: Int -> Int)
% (id :: Int -> Int) :: Int -> Int
% \end{hslisting}
% \pause
% {\Large \faQuestion} Каков \textit{наиболее общий тип} для \inline=id=?
% \end{frame}
% 
% \begin{frame}[fragile]{Примеры типов}
% Тип, который не населен
% \begin{hslisting}
% Prelude> :i Data.Void.Void
% data Data.Void.Void        -- Defined in 'Data.Void'
% \end{hslisting}
% 
% Тип, у которого только один житель
% \begin{hslisting}
% Prelude> :i ()
% data () = ()               -- Defined in 'GHC.Tuple'
% Prelude> :t ()
% () :: ()
% \end{hslisting}
% \end{frame}
% 
% \begin{frame}[fragile]{Как ведет себя функция с типом}
% 
% {\Large \faQuestion} \hsinline=Void -> Int= \\
% \vspace{0.5cm}
% {\Large \faQuestion} \hsinline=()  -> Int= \\
% \vspace{0.5cm}
% {\large \faQuestion} \hsinline=a -> ()= \\
% \vspace{0.5cm}
% \pause
% Ответ:
% \begin{hslisting}
% unit x = ()
% {- §а лучше§ -}
% unit _ = ()
% \end{hslisting}
% \end{frame}
% 
% \begin{frame}[fragile]{Тип пары (Декартово произведение)}
% \begin{center}
% \begin{tikzpicture}[commutative diagrams/every diagram]
% \node (P1) at (90:0cm) {\inline=(a,b)=};
% \node (P2) at (90+ 90+35:2cm) {\inline=a=};
% \node (P3) at (90+270-35:2cm) {\inline=b=};
% \path[commutative diagrams/.cd, every arrow, every label]
% (P1) edge node[swap] {\inline=fst=} (P2)
% (P1) edge node[swap] {\inline=snd=} (P3)
% ;
% \end{tikzpicture}
% \end{center}
% \pause
% Определим \textit{проекции}:
% \begin{hslisting}
% Prelude> let fst (x,_) = x
% Prelude> :t fst
% fst :: (a, b) -> a
% 
% Prelude> let snd (_,y) = y
% Prelude> :t snd
% snd :: (a, b) -> b
% \end{hslisting}
% 
% \end{frame}
% 
% \begin{frame}[fragile]{Произведение (product) -- обобщение пары}
% \begin{center}
% \begin{tikzpicture}[commutative diagrams/every diagram]
% \node (P0) at (90:2.3cm) {\inline=product?=};
% \node (P1) at (90:0cm) {$(a,b)$};
% \node (P2) at (90+ 90+35:3cm) {\inline=a=};
% \node (P3) at (90+270-35:3cm) {\inline=b=};
% \path[commutative diagrams/.cd, every arrow, every label]
% (P0) edge node[swap] {\mbox{\Large $\exists!$} \inline=f=} (P1)
% (P1) edge node[swap] {\inline=fst=} (P2)
% (P1) edge node[swap] {\inline=snd=} (P3)
% (P0) edge[bend right] node[swap] {\inline=fst2=} (P2)
% (P0) edge[bend left]  node[swap] {\inline=snd2=} (P3)
% ;
% \end{tikzpicture}
% \end{center}
% \begin{hslisting}
% fst2 = fst . f
% snd2 = snd . f
% \end{hslisting}
% \end{frame}
% 
% % https://repl.it/languages/haskell
% 
% \begin{frame}[fragile]{Coproduct или Either или тип-сумма}
% \begin{center}
% \begin{tikzpicture}[commutative diagrams/every diagram]
% \node (P0) at (270:1.5cm) {\inline=coproduct?=};
% \node (P1) at (90:0cm) {\inline=a+b=};
% \node (P2) at (165:3cm) {\inline=a=};
% \node (P3) at (15:3cm)  {\inline=b=};
% \path[commutative diagrams/.cd, every arrow, every label]
% (P1) edge node[swap] {\mbox{\Large $\exists!$} \inline=g=} (P0)
% (P2) edge node[swap] {\inline=left= } (P1)
% (P3) edge node[swap] {\inline=right=} (P1)
% (P2) edge[bend right] node[swap] {\inline=l2=} (P0)
% (P3) edge[bend left]  node[swap] {\inline=r2=} (P0)
% ;
% \end{tikzpicture}
% \end{center}
% \begin{hslisting}
% l2 = g . left
% r2 = g . right
% 
% Prelude> :i Either
% data Either a b = Left a | Right b      
% -- Defined in 'Data.Either'
% \end{hslisting}
% \end{frame}
% 
% \begin{frame}[fragile]{Упражнения про произведение и сумму}
% \begin{itemize}
%  \item Придумайте (нагуглите) какой-нибудь тип, который похож на произведение? Реализуйте для него функцию $f$.
%  \item Придумайте (нагуглите) какой-нибудь тип, который похож на копроизведение? Реализуйте для него функцию $g$.
%  \item Кто населяет тип \hsinline=(Void, a)= для произвольного типа \hsinline=a=?
%  \item Кто населяет тип \hsinline=((), a)= для произвольного типа \hsinline=a=?
%  \item Кто населяет тип \hsinline=Either Void a= для произвольного типа \hsinline=a=?
%  \item Кто населяет тип \myinline=Either ()   a= для произвольного типа \hsinline=a=?
%  \item В некоторых упражнениях выше можно строить изоморфизм между двумя типами. Сообразите между какими и постройте там, где возможно.
% \end{itemize}
% 
% \end{frame}
% 
% \begin{frame}[fragile]{Алгебраические типы данных (1/2)}
% Синтаксис
%  \begin{lstlisting}[mathescape=true,language=haskell]
%   data TypeName $arg_1$ $arg_2$ ... $arg_k$ = 
%     | $C_1$ $t_{11}$ $t_{12}$ ... $t_{1n_1}$ 
%     | $C_2$ $t_{21}$ $t_{22}$ ... $t_{2n_2}$ 
%     | ...
%     | $C_m$ $t_{m1}$ $t_{m2}$ ... $t_{mn_m}$ 
%  \end{lstlisting}
%  Примеры
%  \begin{itemize}
%   \item \hsinline!data Bool    = True | False!
%   \item \hsinline!data Status1 = On   | Off!
%   \item \hsinline!data Maybe a = Just a | Nothing!
%   \item \hsinline!data Either a b = Left a | Right b!
%  \end{itemize}
% \end{frame}
% 
% \begin{frame}[fragile]{Алгебраические типы данных (2/2)}
% 
% {\Large \faQuestion} Что такое \textit{связный список}?\\ \pause
% {\Large \faQuestion} А что такое \textit{список} вообще?\\ \pause
% \vspace{1cm}
% Ещё примеры:
% \begin{itemize}
%   \item \hsinline!data [] a = [] | a : [a]!
%   \item \hsinline!data ListG a r = Nil | Cons a r!
%   \item \hsinline!data Fix f = Fix (f (Fix f))! 
% \end{itemize}
% \vspace{1cm}
% Упражнение: из типов \hsinline=ListG= и \hsinline=Fix= можно соорудить что-то похожее на стандартный список Haskell. Предъявите изоморфизм.
% \end{frame}
% 
% \begin{frame}[fragile]{Формальные аксиомы}
% Distinctness: $C_{j}(x) \neq C_{i}(y)$, если $j \neq i$\\
% % https://hal.inria.fr/hal-01212585/document
% \vspace{1cm}
% Injectivity:
% $C_{ij}(x_1,...,x_{n_{ij}}) = C_{ij}(y_1,...,y_{n_{ij}}) \Rightarrow x_k = y_k$\\
% \vspace{1cm}
% Exhaustiveness: \hsinline=x :: ADT= 
% $\Rightarrow \exists i, n$: $x = C_i(y_1,...,y_n)$\\
% \vspace{1cm}
% Selection:
% $s^k_{ij}(C_{ij}(x_1,...,x_{n_{ij}}) = x$
% \end{frame}
% 
% \begin{frame}
% ~\ Упражнение. Для типов, указанных выше, можно построить изоморфные к ним, используя только типы \inline=()=, \inline=Void=, \inline=Either a b=, \inline=(a,b)= (ну и ещё либо рекурсию, либо \inline=Fix=). Попробуйте описать эти типы, предъявляя изоморфизм.
% \end{frame}
% 
% 
% \begin{frame}[fragile]{Как может работать функция со следующим типом?}
% \begin{itemize}
%  \item [{\Large \faQuestion}] \hsinline=[a] -> [a]=
%  \item [{\Large \faQuestion}] \hsinline=[a] -> Bool=
%  \item [{\Large \faQuestion}] \hsinline=(a -> b) -> [a] -> [b]=
%  \item [{\Large \faQuestion}] \hsinline=(a -> a -> Bool) -> [a] -> [a]=
%  \item [{\Large \faQuestion}] \hsinline=(a -> a -> Ordering) -> [a] -> [a]=, если
% \begin{hslisting}
% Prelude> :i Ordering
% data Ordering = LT | EQ | GT   
% \end{hslisting}
%  \item [{\Large \faQuestion}] \hsinline=(a -> b -> a) -> a -> [b] -> a=
%  \item [{\Large \faQuestion}] \hsinline=Maybe a -> Maybe b -> (a -> b -> c) -> Maybe c=
% \end{itemize}
% \end{frame}


\end{document}
