\documentclass[
  xcolor={svgnames},
  hyperref={colorlinks,citecolor=DeepPink4,linkcolor=DarkRed,urlcolor=DarkBlue}]{beamer}
\usetheme{CambridgeUS}
\beamertemplatenavigationsymbolsempty % remove navigation bar

\usepackage{xcolor}
\definecolor{YellowGreen} {HTML}{B5C28C}
\definecolor{ForestGreen} {HTML}{009B55}
\definecolor{MyBackground}{HTML}{F0EDAA}


\usepackage{xltxtra} % load xunicode
\usepackage{polyglossia}
\setmainlanguage{russian}
\setotherlanguage{english}

\let\cyrillicfonttt\monofamily
\newfontfamily\dejaVuSansMono{DejaVu Sans Mono}
% https://github.com/vjpr/monaco-bold/raw/master/MonacoB/MonacoB.otf
\newfontfamily\monacoB{MonacoB} 

\usepackage{listings}
\lstdefinestyle{style1}{
  language=haskell,
  numbers=left,
  stepnumber=1,
  numbersep=10pt,
  tabsize=4,
  showspaces=false,
  showstringspaces=false
}
\lstdefinestyle{hsstyle1}
         { language=haskell
         , basicstyle=\monacoB
         , deletekeywords={Int,Float,String,List,Void}
         , breaklines=true
         , columns=fullflexible
         % , commentstyle=\color{ForestGreen}
%          , escapeinside=§§
%          , escapebegin=\begin{russian}\commentfont
%          , escapeend=\end{russian}
         % , commentstyle=\color{ForestGreen}
         , escapeinside=§§
         , escapebegin=\begin{russian}\monacoB\color{ForestGreen}
         , escapeend=\end{russian}
         , mathescape=true
%          , backgroundcolor = \color{MyBackground}
}

\newcommand{\inline}[1]{\lstinline{haskell}{#1}}
\def\hsinline{\lstinline[style={hsstyle1}]}
\def\inline{\hsinline}
\def\myinline{\lstinline[basicstyle=\monacoB]}
% \renewcommand{\vskip}{\vspace{1cm}}

\lstnewenvironment{hslisting} {
    \lstset { style={hsstyle1} }
  }
  {}
  
%%%%%%%%%%%%%%%%%%%%%%%%%%%%%%%%%%%%%%%%%%%%%%%%%%%%%%%%%%  
\setmainfont[
 Ligatures=TeX,
 Extension=.otf,
 BoldFont=cmunbx,
 ItalicFont=cmunti,
 BoldItalicFont=cmunbi,
]{cmunrm}
% С засечками (для заголовков)
\setsansfont[
 Ligatures=TeX,
 Extension=.otf,
 BoldFont=cmunsx,
 ItalicFont=cmunsi,
]{cmunss}
% \setmonofont[Scale=0.6]{Monaco}

\usefonttheme{professionalfonts}
\usepackage{times}
\usepackage{tikz}
\usetikzlibrary{cd}
% \usepackage{tikz-cd}
\usepackage{amsmath}
%\DeclareMathOperator{->}{\rightarrow}
\newcommand\iso{\ensuremath{\cong}}
\usepackage{verbatim}
\usepackage{graphicx}
\usetikzlibrary{arrows,shapes}

\usepackage{fontawesome}
% \newfontfamily{\FA}{Font Awesome 5 Free} % some glyphs missing
\expandafter\def\csname faicon@facebook\endcsname{{\FA\symbol{"F09A}}}
\def\faQuestionSign{{\FA\symbol{"F059}}}
\def\faQuestion{{\FA\symbol{"F128}}}
\def\faExclamation{{\FA\symbol{"F12A}}}
\def\faUploadAlt{{\FA\symbol{"F093}}}
\def\faLemon{{\FA\symbol{"F094}}}
\def\faPhone{{\FA\symbol{"F095}}}
\def\faCheckEmpty{{\FA\symbol{"F096}}}
\def\faBookmarkEmpty{{\FA\symbol{"F097}}}

\newcommand{\faGood}{\textcolor{ForestGreen}{\faThumbsUp}}
\newcommand{\faBad}{\textcolor{red}{\faThumbsODown}}

\usepackage{soul} % for \st that strikes through
\usepackage[normalem]{ulem} % \sout

% \usepackage{minted}
% \newcommand{\inline}[1]{\mintinline{haskell}{#1}}


\title[Занятие 2]{Пользовательские типы данных}
\author{Косарев Дмитрий a.k.a. Kakadu}

\institute{матмех СПбГУ}

\date{\today}

\AtBeginSection[]
{
  \begin{frame}<beamer>
    \frametitle{Outline}
    \tableofcontents[currentsection,currentsubsection]
  \end{frame}
}

\newcommand{\verbatimfont}[1]{\def\verbatim@font{#1}}
\usepackage{verbatimbox}
\begin{document}
\maketitle

% For every picture that defines or uses external nodes, you'll have to
% apply the 'remember picture' style. To avoid some typing, we'll apply
% the style to all pictures.
\tikzstyle{every picture}+=[remember picture]

% By default all math in TikZ nodes are set in inline mode. Change this to
% displaystyle so that we don't get small fractions.
\everymath{\displaystyle}

% Uncomment these lines for an automatically generated outline.
% \begin{frame}{Outline}
%   \tableofcontents
% \end{frame}

\begin{frame}[fragile]{Натуральные числа в стиле Пеано}
Положим у нас есть ``ноль'' (ну или ``один'') и есть ``следующий''.
\pause
\begin{hslisting}
data Nat = Zero | Succ Nat
\end{hslisting}
\pause
Простые упражнения: сложить или умножить пару чисел Пеано.
\end{frame}

\begin{frame}[fragile]{JSON}
\begin{itemize}
 \item true и false
 \item числа
 \item строки
 \item null
 \item массивы
 \item Объекты как набор пар ``ключ-значение''
\end{itemize}
\pause
\begin{verbnobox}[\monacoB]
data JSON = 
            JNull
          | JBool Bool
          | JNum   Float
          | JStr   String
          | JArray [JSON]
          | JObj   [(String, JSON)]
\end{verbnobox}
\end{frame}

\begin{frame}[fragile]{Простой пример: арифметика (1/2)}
\begin{verbnobox}[\monacoB]
data Expr =
    EConst Int
  | EMul Expr Expr
  | EAdd Expr Expr
  deriving Show
\end{verbnobox}
\begin{verbnobox}[\monacoB]
eval :: Expr -> Int
\end{verbnobox}
\pause
\begin{verbnobox}[\monacoB]
eval (EConst n) = n
eval (EAdd l r) = (eval l) + (eval r)
eval (EMul l r) = (eval l) * (eval r)
\end{verbnobox}
\end{frame}

\begin{frame}[fragile]{Простой пример: арифметика (2/2)}
\begin{verbnobox}[\monacoB]
e1 :: Expr
e1 = EMul (EAdd (EConst 1) (EConst 2)) (EConst 3)

e2 = (1 + 2) * 3
\end{verbnobox}
\pause
\begin{verbnobox}[\monacoB]
*Main> :l Arith
*Main> eval e1
9
*Main> e1
EMul (EAdd (EConst 1) (EConst 2)) (EConst 3)
\end{verbnobox}
\vspace{1cm}
\textit{Deep embedding} vs. \textit{shallow embedding}
\end{frame}

\begin{frame}[fragile]{Арифметика с переменными}
\begin{verbnobox}[\monacoB]
data Expr = 
    EConst Int
  | EMul Expr Expr
  | EAdd Expr Expr
  | EVar String
 
eval :: [(String,Int)] -> Expr -> Maybe Int
\end{verbnobox}
\pause
\begin{verbnobox}[\monacoB]
eval _ (EConst n) = Just n
eval env (EAdd l r) = 
  Just ((eval env l) + (eval env r))
eval env (EMul l r) = 
  Just ((eval env l) * (eval env r))
eval env (EVar s) = lookup s env 
\end{verbnobox}
\end{frame}

\begin{frame}[fragile]{Содержательный пример (1/3)}
\begin{verbnobox}[\monacoB]
import Data.Word
data InetAddr = InetAddr Word8 Word8 Word8 Word8

data ConnectionState = 
  Connecting  | Connected  | Disconnected 
  
data ConnectionInfo = CInfo 
  { state ::                   ConnectionState
  , server ::                  InetAddr
  , last_ping_time ::          Maybe Time
  , last_ping_id ::            Maybe Int
  , session_id ::              Maybe String
  , when_initiated ::          Maybe Time
  , when_disconnected ::       Maybe Time
  } 
\end{verbnobox}
\end{frame}

\begin{frame}[fragile]{Содержательный пример (2/3)}
\begin{verbnobox}[\monacoB]
data Connecting = 
  Connecting { when_initiated:: Time } 
data Connected  = Connected 
  { last_ping  :: Maybe (Time,Int)
  , session_id :: String } 
data Disconnected = 
  Disconnected { when_disconnected :: Time } 
 
data ConnectionState = 
    SConnecting   Connecting 
  | SConnected    Connected 
  | SDisconnected Disconnected 
\end{verbnobox}
\end{frame}

\begin{frame}[fragile]{Содержательный пример (3/3)}
\begin{verbnobox}[\monacoB]
data ConnectionState = 
    SConnecting   Connecting 
  | SConnected    Connected 
  | SDisconnected Disconnected 
 
data ConnectionInfo = Cinfo 
  { state ::  ConnectionState
  , server :: InetAddr } 
\end{verbnobox}
\pause
Лозунг: ``плохие'' состояния (значения) должны быть непредставимы в типах.
\end{frame}

\begin{frame}[fragile]{Пример про почту (1/2)}
\begin{verbnobox}[\monacoB]
data Contact = Contact 
    { name :: Name
    , emailContactInfo :: EmailContactInfo
    , postalContactInfo :: PostalContactInfo }
\end{verbnobox}
\pause
Хочется, чтобы у контакта был \textit{хотя бы один} адрес: либо электронной, либо физической почты.
\pause
Что вы думаете о вот таком?
\begin{hslisting}
data Contact = Contact 
    { name :: Name
    , emailContactInfo :: Maybe EmailContactInfo
    , postalContactInfo :: Maybe PostalContactInfo }
\end{hslisting}
\end{frame}

\begin{frame}[fragile]{Пример про почту (2/2)}
\begin{verbnobox}[\monacoB]
data ContactInfo = 
    EmailOnly    EmailContactInfo
  | PostOnly     PostalContactInfo
  | EmailAndPost EmailContactInfo PostalContactInfo

data Contact = Contact
  { name        :: Name
  , contactInfo :: ContactInfo
  }
\end{verbnobox}

Если, посмотрев на тип, сразу понятно какие состояния корректные, а какие нет, то это считает хорошим дизайном.\\
\vspace{0.5cm}
Пример взят \href{https://fsharpforfunandprofit.com/posts/designing-with-types-making-illegal-states-unrepresentable/}{отсюда.}
\end{frame}

\begin{frame}[fragile]{Пример про длины (1/2)}
\hsinline={- Measures.hs -}=
\begin{verbnobox}[\monacoB]
module Measures (Miles(), Kilometers, miles
  , kilometers, addKilometers, addMiles) where

data Miles = Mile Int
data Kilometers = KM Int
\end{verbnobox}
\begin{verbnobox}[\monacoB]
miles = Mile
kilometers x = KM x
\end{verbnobox}
\begin{verbnobox}[\monacoB]
addMiles :: Miles -> Miles -> Miles 
addMiles (Mile x) (Mile y) = Mile (x+y)
\end{verbnobox}
\begin{verbnobox}[\monacoB]
addKilometers :: Kilometers -> Kilometers -> 
                 Kilometers
addKilometers (KM x) (KM y) = KM (x+y)
\end{verbnobox}
\end{frame}

\begin{frame}[fragile]{Пример про длины (2/2) }
\hsinline={- Measures.hs -}=
\begin{verbnobox}[\monacoB]
import Measures

x = (miles 1) `addMiles` (miles 2)
\end{verbnobox}
\pause
\begin{verbnobox}[\monacoB]
y :: Miles
y = 1 `addMiles` (miles 2)
{-
  Long error message which requires a notion of 
  type class to be understandable
  |              
8 | y = 1 `addMiles` (miles 2)
  |     ^
-}
\end{verbnobox}
Здесь запрещено случайно складывать числа с милями, а мили с километрами.
\end{frame}

\begin{frame}[fragile]{Домашнее упражнение про длины}
%\hsinline={- Measures.hs -}=
Сейчас у нас отдельная функция сложения для миль и отдельная километров. Сделайте одну функцию, которая берет две ``длины в одинаковой системе'' и выдает ``длину в такой же системе''.
\vspace{1cm}

Вам пригодятся знания:
\begin{itemize}
 \item код выше можно переиспользовать немного поменяв;
 \item тип \hsinline=Void= можно описать вручную, он не встроенный;
 \item типовые параметры могут использоваться в правой части, а могут и не использоваться;
 \item \href{https://www.kinopoisk.ru/name/204813/}{фильмография} Луи де Фюнеса может подсказать какой термин из функционального программирования надо гуглить.
\end{itemize}
\end{frame}

\end{document}
